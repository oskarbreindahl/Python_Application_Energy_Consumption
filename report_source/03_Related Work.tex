\fancyhead[R]{Python Application Energy Consumption}
\section{Related Work}
\label{sec:relatedwork}
There are several studies attempting to compare the energy efficiency of different programming languages. In these studies Python ranks in the bottom in terms of performance and energy consumption, as presented by Georgiou et al.\cite{georgiou2017analyzing}, Pereira et al.\cite{pereira_rank_efficiency, pereira2017energy} and Koedijk and Oprescu\cite{koedijk2022finding}. Methods of measuring energy consumption in these studies range from external monitoring tools, akin to the one used in this project, to software-based ones that query functions provided by the onboard processor. As a counterpoint to these studies, van Kempen et al.\cite{van2024s} finds that these comparisons between languages do not prove a causal relationship between choice of programming language and energy consumption. This is partly because they fail to account for differences in language and benchmark implementation. This supports the approach of this thesis in comparing different implementations of the language using a uniform benchmark.

The effects of hardware on energy efficiency are also documented in research. For example, Lopez-Novoa\cite{lopez2019exploring} benchmarks the performance and energy efficiency of different generations of Intel processors. Interestingly, they find that while the energy consumption is lower for newer generations, the improvement is not proportional to the performance increase.

Python-specific energy research includes more targeted studies that explore particular components of the interpreter. Lamprakos et al.\cite{lamprakos_energy} investigate the energy implications of dynamic memory allocation within CPython, showing that modifications to memory handling strategies can result in substantial energy savings. Similarly, Holm et al.\cite{holm2020gpu} explore Python’s viability as a high-level language for GPU-based computation, concluding that Python can match the energy efficiency of C++ in certain scenarios due to efficient library bindings and reduced development overhead. Reya et al.\cite{reya2023greenpy} look at the effects of specific coding practices on Python energy efficiency. They find that practices such as string concatenation over string formatting or assignment operator over data initialization offer improvements of up to 2.2 times in energy efficiency. 

Pfeiffer's paper\cite{pfeiffer2024energy}, which is the foundation of this thesis, compares the energy consumption of different versions of Python in an experimental setup very similar to the one used in this project. The paper finds that, in the given setup, newer versions of Python are more energy efficient. Much like Pfeiffer, this thesis focuses on the impact of changing the execution environment rather than application code. Building on Pfeiffer's paper, we provide a broader basis for comparison, while switching to a synthetic rather than real-world benchmark, thus extending rather than repeating Pfeiffer's research.

Pfeiffer’s work and this thesis also fit within broader green software initiatives such as green coding, which optimizes application source code \cite{radersma2022green,lamprakos_energy,holm2020gpu,reya2023greenpy}, green software engineering (GSE), which targets team practices \cite{lorincz2019greener,freed2023investigation,roque2025unveiling}, and the overarching field of green computing, which addresses infrastructure and policy \cite{paul2023comprehensive,ahmad2023green,tiwari2021review}. This thesis fits at the intersection of GSE and green computing. It bridges GSE’s practice-driven approach with green computing's infrastructure-level focus and contributes a practical approach to analyse application energy consumption.
