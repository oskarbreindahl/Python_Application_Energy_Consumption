\newpage
\fancyhead[R]{Python Application Energy Consumption}
\section*{Abstract}
\label{sec:abstract}

Python is one of the most used programming languages in the world, and while it is one of the least performant languages, its developers actively work on increasing its performance. However, there are currently no efforts by Python’s developers to decrease its energy consumption, making it hard for developers using the language to make informed decisions in this regard when developing applications. Additionally, no standard approach or framework exists to compare and benchmark Python energy consumption across versions, operating systems and hardware. In this thesis, we establish a replicable experiment design for comparing the energy consumption of different Python versions across different operating systems and hardware setups. Additionally, we provide initial data from running this experiment on five different Python versions across four different Unix-based operating systems and two different hardware setups. These contributions aim to establish a method of benchmarking Python energy consumption while providing a basis of comparison for developers looking to use Python for application development. We find that choice of Python version, operating system and hardware all have a statistically significant impact on energy consumption.