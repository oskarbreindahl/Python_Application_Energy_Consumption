% This is samplepaper.tex, a sample chapter demonstrating the
% LLNCS macro package for Springer Computer Science proceedings;
% Version 2.21 of 2022/01/12
%
\documentclass[runningheads]{llncs}
%
\usepackage[T1]{fontenc}
% T1 fonts will be used to generate the final print and online PDFs,
% so please use T1 fonts in your manuscript whenever possible.
% Other font encondings may result in incorrect characters.
%
\usepackage{graphicx}
% Used for displaying a sample figure. If possible, figure files should
% be included in EPS format.
%
% If you use the hyperref package, please uncomment the following two lines
% to display URLs in blue roman font according to Springer's eBook style:
%\usepackage{color}
%\renewcommand\UrlFont{\color{blue}\rmfamily}
%\urlstyle{rm}
\usepackage{hyperref}
\usepackage[acronym]{glossaries}
\glsdisablehyper
\usepackage{todonotes}
\usepackage{subcaption}
\usepackage{caption}
\captionsetup[table]{name=Tab.}
\usepackage{xspace}
\usepackage{siunitx}
\usepackage{multicol}
\usepackage{multirow}
\usepackage{booktabs}
\usepackage{svg}
\usepackage{xurl}
\usepackage[sort,compress]{cite}

% \usepackage{draftwatermark}
% \SetWatermarkText{DRAFT}
% \SetWatermarkScale{5}
% \usepackage{framed}

% I inserted this since I always got a numbered but blank first page
% I do not know why
% I followed this advise: https://tex.stackexchange.com/a/434340
\usepackage{atbegshi}
\AtBeginDocument{\AtBeginShipoutNext{\AtBeginShipoutDiscard}\addtocounter{page}{-1}}



\newcommand{\citeurl}[1]{\unskip\footnote{{\scriptsize \url{#1}}}}
\newcommand{\pkgname}[1]{{\scriptsize \textsf{#1}}}
\newcommand{\projname}[1]{{\small \textsf{#1}}}
% \newcommand{\toolname}[1]{{\scriptsize \texttt{#1}}}
\newcommand{\toolname}[1]{{\small \texttt{#1}}}

\newcommand{\python}{\projname{Python}\xspace}
\newcommand{\cp}{\projname{CPython}\xspace}
\newcommand{\cpv}[1]{\projname{CPython 3.{#1}}\xspace}
\newcommand{\flask}{\projname{Flask}\xspace}
\newcommand{\django}{\projname{Django}\xspace}
\newcommand{\rust}{\projname{Rust}\xspace}
\newcommand{\java}{\projname{Java}\xspace}
\newcommand{\cc}{\projname{C\nolinebreak\hspace{-.05em}\raisebox{.4ex}{\tiny +}\nolinebreak\hspace{-.10em}\raisebox{.4ex}{\tiny +}}\xspace}

\renewcommand{\tableautorefname}{Tab.}
\renewcommand{\figureautorefname}{Fig.}
\renewcommand{\sectionautorefname}{Sec.}
\renewcommand{\subsectionautorefname}{Sec.}
\renewcommand{\subsubsectionautorefname}{Sec.}

\newacronym{pep}{PEP}{Python Enhancement Proposal}
\newacronym{watt}{W}{Watt}
\newacronym{joule}{J}{Joule}
\newacronym{sut}{SUT}{System Under Test}
\newacronym{pypi}{PyPI}{Python Packaging Index}
\newacronym{pyperformance}{\projname{pyperformance}}{The Python Performance Benchmark Suite}
\newacronym{os}{OS}{Operating System}
\newacronym{cpu}{CPU}{Central Processing Unit}
\newacronym{rpi}{RPI}{Raspberry Pi}

\newcommand{\rpi}[1]{\toolname{Raspberri Pi #1}\xspace}
\newcommand{\rpia}{\gls{rpi}\xspace}

%
% Page limit: 12 to 16 pages
\begin{document}
%
\title{Energy Consumption of Python Performance Benchmarks Depends on Execution Environments}
%
%\titlerunning{Abbreviated paper title}
% If the paper title is too long for the running head, you can set
% an abbreviated paper title here
%
\author{Oskar Emil Breindahl\inst{1}
\and
Rolf-Helge Pfeiffer\inst{1}\orcidID{0000-0003-2585-6473}
%\and
%Third Author\inst{3}\orcidID{2222--3333-4444-5555}
}
%
\authorrunning{O. Breindahl, R.-H. Pfeiffer}\
%\authorrunning{Fst. Author}\
% First names are abbreviated in the running head.
% If there are more than two authors, 'et al.' is used.
%
\institute{IT University of Copenhagen, Rued Langgaards Vej 7, 2300 Copenhagen
\email{\{osbr,ropf\}@itu.dk}}
% \institute{A University, Street, City
% \email{author@university.edu}}
%\url{http://www.springer.com/gp/computer-science/lncs} \and
%ABC Institute, Rupert-Karls-University Heidelberg, Heidelberg, Germany\\
%\email{\{abc,lncs\}@uni-heidelberg.de}}
%
\maketitle              % typeset the header of the contribution
%
\begin{abstract}
% The abstract should briefly summarize the contents of the paper in 150--250 words.
\python is one of the most used programming languages in the world.
Developers of the \python interpreter increase its performance over recent releases and measure these with \gls{pyperformance}.
However, little is known about the impact of performance increases on the energy consumption of \python programs when executed in different environments like operating systems or processors.
In this paper, we study via a controlled lab experiment the energy consumption of benchmarks from \gls{pyperformance} when executed on five versions of the \python interpreter \cp, on four different operating systems, and on two different processors.
Our results indicate that executing Python programs on a Cortex-A72 processor running Manjaro Linux and \cpv{12} is XX times more energy efficient than on a Cortex-A53 running FreeBSD with \cpv{10}.
Even on the same processor, energy consumption of \python programs decreases by up to XX\% when executed on Manjaro Linux and a newer versions of \python, compared to other operating systems and older versions of \cp.
\keywords{Software engineering \and Energy consumption \and CPython.}
\end{abstract}

\section{Introduction}\label{sec:introduction}

The responsibility of software developers to create sustainable and energy-efficient software is becoming more and more apparent\cite{caballar2024we} and the environmental impact and sustainability of software is increasingly studied.\cite{ahmad2023green,caballar2024we,freed2023investigation,gupta2021chasing,adersma2022green,lamprakos_energy,holm2020gpu,reya2023greenpy,lorincz2019greener,freed2023investigation,roque2025unveiling,paul2023comprehensive,ahmad2023green,tiwari2021review}\todo{Too much sausage?}.

\python is likely the most popular~\cite{djurdjev2024popularity,pypl,tiobe} and most used~\cite{stackover, statista} programming language in the world.
Big corporations create software products with it.
For example, Google’s \projname{YouTube} is powered by \python~\cite{winters2020software}, circa 20\% of \projname{Facebook}’s infrastructure is written in \python~\cite{komorn2016python}, \projname{Instagram} is a \python application~\cite{ni2016web}, or circa 80\% of \projname{Spotify}’s backend services are written in \python~\cite{van2013how}.

Like other dynamically typed and interpreted programming languages, running certain \python programs is reported to be slow and of low energy efficiency~\cite{pereira_rank_efficiency}.
In recent years, the developers of \cp, the reference implementation of the \python interpreter, continuously increase the performance of \cp, i.e., they increase execution times of \python programs.

Since energy consumption of software is influenced by execution times ($E = P \times t$) and since \python is a comparably slow dynamically typed interpreted language, it might be worthwhile to continue to increase performance of \cp over the coming releases.
We believe that additionally focusing on the energy consumption of programs executed on \cp should be in focus as well.\todo{Improve this.}
Currently, there is no \gls{pep} to reduce energy consumption of \python programs as there are \glspl{pep} to increase performance of \cp~\cite{pep_index}.

In this paper, we present a light weight experiment design that allows to compare the energy consumption of \python programs that are executed on various configurations of hardware, \gls{os}, and \python versions.
Our goals are \emph{a)} to investigate if certain configurations of \glspl{cpu}, \glspl{os}, and \python versions significantly impact the energy consumption of \python benchmarks from \gls{pyperformance}, and \emph{b)} to provide an experiment design that is so light weight that \cp developers can efficiently include it in their current setup when running \gls{pyperformance}.

We investigate the following three research questions:
\begin{itemize}
  \item \textbf{RQ1:} \textit{How do different \glspl{cpu} impact the energy consumption of \gls{pyperformance} benchmarks?}
  \item \textbf{RQ2:} \textit{How do different \glspl{os} impact the energy consumption of \gls{pyperformance} benchmarks?}
  \item \textbf{RQ3:} \textit{How do different versions of \cp impact the energy consumption of \gls{pyperformance} benchmarks?}
\end{itemize}

We measure energy consumption of a \gls{sut} with an \toolname{Otii Ace Pro}~\cite{qoitech2022otii}.
In this paper, \glspl{sut} are a \toolname{Raspberry Pi 3B+} (ARM-Cortex-A53) and \toolname{Raspberry Pi 4B} (ARM-Cortex A72) respectively.
% TODO: Add these to references
% https://datasheets.raspberrypi.com/rpi3/raspberry-pi-3-b-plus-product-brief.pdf
% https://datasheets.raspberrypi.com/rpi4/raspberry-pi-4-datasheet.pdf
On each \toolname{Raspberry Pi}, we sequentially deploy one of the four Unix(-like) \glspl{os} \projname{Alpine}, \projname{Manjaro}, \projname{Ubuntu}, and \projname{FreeBSD}.
On each \gls{os}, we sequentially deploy different versions of \cp.
The combination of \gls{cpu}, \gls{os}, and version of \cp is a configuration.
In our experiment, we investigate 40 different configurations.
For each configuration, we execute \gls{pyperformance} and measure execution times of various benchmarks and their power draw during execution.

We find that all three variables of \gls{cpu}, \gls{os}, and version of \cp have a statistically significant impact on energy consumption during benchmark execution.
\todo{Fill in high level results here once the paper is stable}
% Fill in high level results here!


The contributions of this paper are:
\begin{itemize}
  \item We present an experiment design to directly measure the energy consumption of 40 different configurations of \gls{cpu}, \gls{os}, and version of \cp.
  \item We demonstrate that benchmarks executed on certain configurations of \gls{cpu}, \gls{os}, and version of \cp consume significantly less energy compared to other widely used configurations.
  \item We provide a replication kit\footnote{\url{https://github.com/oskarbreindahl/Python_Application_Energy_Consumption}} that allows to automatically replicate our results. It also contains all results reported in this paper and allows for reproduction.\todo{Move repo and anonymize link for review.}
\end{itemize}


\bibliographystyle{splncs04}
\bibliography{bibliography}
%



\end{document}
