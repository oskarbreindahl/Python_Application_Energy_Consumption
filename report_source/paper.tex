% This is samplepaper.tex, a sample chapter demonstrating the
% LLNCS macro package for Springer Computer Science proceedings;
% Version 2.21 of 2022/01/12
%
\documentclass[runningheads]{llncs}
%
\usepackage[T1]{fontenc}
% T1 fonts will be used to generate the final print and online PDFs,
% so please use T1 fonts in your manuscript whenever possible.
% Other font encondings may result in incorrect characters.
%
\usepackage{graphicx}
% Used for displaying a sample figure. If possible, figure files should
% be included in EPS format.
%
% If you use the hyperref package, please uncomment the following two lines
% to display URLs in blue roman font according to Springer's eBook style:
%\usepackage{color}
%\renewcommand\UrlFont{\color{blue}\rmfamily}
%\urlstyle{rm}
\usepackage{hyperref}
\usepackage[acronym]{glossaries}
\glsdisablehyper
\usepackage{todonotes}
\usepackage{subcaption}
\usepackage{caption}
\captionsetup[table]{name=Tab.}
\usepackage{xspace}
\usepackage{siunitx}
\usepackage{multicol}
\usepackage{multirow}
\usepackage{booktabs}
\usepackage{svg}
\usepackage{xurl}
\usepackage[sort,compress]{cite}

% \usepackage{draftwatermark}
% \SetWatermarkText{DRAFT}
% \SetWatermarkScale{5}
% \usepackage{framed}

% I inserted this since I always got a numbered but blank first page
% I do not know why
% I followed this advise: https://tex.stackexchange.com/a/434340
\usepackage{atbegshi}
\AtBeginDocument{\AtBeginShipoutNext{\AtBeginShipoutDiscard}\addtocounter{page}{-1}}



\newcommand{\citeurl}[1]{\unskip\footnote{{\scriptsize \url{#1}}}}
\newcommand{\pkgname}[1]{{\scriptsize \textsf{#1}}}
\newcommand{\projname}[1]{{\small \textsf{#1}}}
% \newcommand{\toolname}[1]{{\scriptsize \texttt{#1}}}
\newcommand{\toolname}[1]{{\small \texttt{#1}}}

\newcommand{\python}{\projname{Python}\xspace}
\newcommand{\cp}{\projname{CPython}\xspace}
\newcommand{\cpv}[1]{\projname{CPython 3.{#1}}\xspace}
\newcommand{\flask}{\projname{Flask}\xspace}
\newcommand{\django}{\projname{Django}\xspace}
\newcommand{\rust}{\projname{Rust}\xspace}
\newcommand{\java}{\projname{Java}\xspace}
\newcommand{\cc}{\projname{C\nolinebreak\hspace{-.05em}\raisebox{.4ex}{\tiny +}\nolinebreak\hspace{-.10em}\raisebox{.4ex}{\tiny +}}\xspace}

\renewcommand{\tableautorefname}{Tab.}
\renewcommand{\figureautorefname}{Fig.}
\renewcommand{\sectionautorefname}{Sec.}
\renewcommand{\subsectionautorefname}{Sec.}
\renewcommand{\subsubsectionautorefname}{Sec.}

\newacronym{pep}{PEP}{Python Enhancement Proposal}
\newacronym{watt}{W}{Watt}
\newacronym{joule}{J}{Joule}
\newacronym{sut}{SUT}{System Under Test}
\newacronym{pypi}{PyPI}{Python Packaging Index}
\newacronym{pyperformance}{\projname{pyperformance}}{The Python Performance Benchmark Suite}

%
% Page limit: 12 to 16 pages
\begin{document}
%
\title{Energy Consumption of Python Performance Benchmarks Depends on Execution Environments}
%
%\titlerunning{Abbreviated paper title}
% If the paper title is too long for the running head, you can set
% an abbreviated paper title here
%
\author{Oskar Emil Breindahl\inst{1}
\and
Rolf-Helge Pfeiffer\inst{1}\orcidID{0000-0003-2585-6473}
%\and
%Third Author\inst{3}\orcidID{2222--3333-4444-5555}
}
%
\authorrunning{O. Breindahl, R.-H. Pfeiffer}\
%\authorrunning{Fst. Author}\
% First names are abbreviated in the running head.
% If there are more than two authors, 'et al.' is used.
%
\institute{IT University of Copenhagen, Rued Langgaards Vej 7, 2300 Copenhagen
\email{\{osbr,ropf\}@itu.dk}}
% \institute{A University, Street, City
% \email{author@university.edu}}
%\url{http://www.springer.com/gp/computer-science/lncs} \and
%ABC Institute, Rupert-Karls-University Heidelberg, Heidelberg, Germany\\
%\email{\{abc,lncs\}@uni-heidelberg.de}}
%
\maketitle              % typeset the header of the contribution
%
\begin{abstract}
% The abstract should briefly summarize the contents of the paper in 150--250 words.
\python is one of the most used programming languages in the world.
Developers of the \python interpreter increase its performance over recent releases and measure these with \gls{pyperformance}.
However, little is known about the impact of performance increases on the energy consumption of \python programs when executed in different environments like operating systems or processors.
In this paper, we study via a controlled lab experiment the energy consumption of benchmarks from \gls{pyperformance} when executed on five versions of the \python interpreter \cp, on four different operating systems, and on two different processors.
Our results indicate that executing Python programs on a Cortex-A72 processor running Manjaro Linux and \cpv{12} is XX times more energy efficient than on a Cortex-A53 running FreeBSD with \cpv{10}.
Even on the same processor, energy consumption of \python programs decreases by up to XX\% when executed on Manjaro Linux and a newer versions of \python, compared to other operating systems and older versions of \cp.
\keywords{Software engineering \and Energy consumption \and CPython.}
\end{abstract}


\bibliographystyle{splncs04}
\bibliography{bibliography}
%



\end{document}
