\begin{abstract}
% The abstract should briefly summarize the contents of the paper in 150--250 words.
\python is one of the most used programming languages in the world.
Developers of the \python interpreter increase its performance over recent releases and measure these with \gls{pyperformance}.
However, little is known about the impact of performance increases on the energy consumption of \python programs when executed in different environments like operating systems or processors.
In this paper, we study via a controlled lab experiment the energy consumption of benchmarks from \gls{pyperformance} when executed on five versions of the \python interpreter \cp, on four different operating systems, and on two different processors.
Our results indicate that executing Python programs on a Cortex-A72 processor running Manjaro Linux and \cpv{12} is XX times more energy efficient than on a Cortex-A53 running FreeBSD with \cpv{10}.
Even on the same processor, energy consumption of \python programs decreases by up to XX\% when executed on Manjaro Linux and a newer versions of \python, compared to other operating systems and older versions of \cp.
\keywords{Software engineering \and Energy consumption \and CPython.}
\end{abstract}
