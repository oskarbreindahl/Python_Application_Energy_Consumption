\section{Introduction}\label{sec:introduction}

The responsibility of software developers to create sustainable and energy-efficient software is becoming more and more apparent\cite{caballar2024we} and the environmental impact and sustainability of software is increasingly studied.\cite{ahmad2023green,caballar2024we,freed2023investigation,gupta2021chasing,adersma2022green,lamprakos_energy,holm2020gpu,reya2023greenpy,lorincz2019greener,freed2023investigation,roque2025unveiling,paul2023comprehensive,ahmad2023green,tiwari2021review}\todo{Too much sausage?}.

\python is likely the most popular~\cite{djurdjev2024popularity,pypl,tiobe} and most used~\cite{stackover, statista} programming language in the world.
Big corporations create software products with it.
For example, Google’s \projname{YouTube} is powered by \python~\cite{winters2020software}, circa 20\% of \projname{Facebook}’s infrastructure is written in \python~\cite{komorn2016python}, \projname{Instagram} is a \python application~\cite{ni2016web}, or circa 80\% of \projname{Spotify}’s backend services are written in \python~\cite{van2013how}.

Like other dynamically typed and interpreted programming languages, running certain \python programs is reported to be slow and of low energy efficiency~\cite{pereira_rank_efficiency}.
In recent years, the developers of \cp, the reference implementation of the \python interpreter, continuously increase the performance of \cp, i.e., they increase execution times of \python programs.

Since energy consumption of software is influenced by execution times ($E = P \times t$) and since \python is a comparably slow dynamically typed interpreted language, it might be worthwhile to continue to increase performance of \cp over the coming releases.
We believe that additionally focusing on the energy consumption of programs executed on \cp should be in focus as well.\todo{Improve this.}
Currently, there is no \gls{pep} to reduce energy consumption of \python programs as there are \glspl{pep} to increase performance of \cp~\cite{pep_index}.

In this paper, we present a light weight experiment design that allows to compare the energy consumption of \python programs that are executed on various configurations of hardware, \gls{os}, and \python versions.
Our goals are \emph{a)} to investigate if certain configurations of \glspl{cpu}, \glspl{os}, and \python versions significantly impact the energy consumption of \python benchmarks from \gls{pyperformance}, and \emph{b)} to provide an experiment design that is so light weight that \cp developers can efficiently include it in their current setup when running \gls{pyperformance}.

We investigate the following three research questions:
\begin{itemize}
  \item \textbf{RQ1:} \textit{How do different \glspl{cpu} impact the energy consumption of \gls{pyperformance} benchmarks?}
  \item \textbf{RQ2:} \textit{How do different \glspl{os} impact the energy consumption of \gls{pyperformance} benchmarks?}
  \item \textbf{RQ3:} \textit{How do different versions of \cp impact the energy consumption of \gls{pyperformance} benchmarks?}
\end{itemize}

We measure energy consumption of a \gls{sut} with an \toolname{Otii Ace Pro}~\cite{qoitech2022otii}.
In this paper, \glspl{sut} are a \toolname{Raspberry Pi 3B+} (ARM-Cortex-A53) and \toolname{Raspberry Pi 4B} (ARM-Cortex A72) respectively.
% TODO: Add these to references
% https://datasheets.raspberrypi.com/rpi3/raspberry-pi-3-b-plus-product-brief.pdf
% https://datasheets.raspberrypi.com/rpi4/raspberry-pi-4-datasheet.pdf
On each \toolname{Raspberry Pi}, we sequentially deploy one of the four Unix(-like) \glspl{os} \projname{Alpine}, \projname{Manjaro}, \projname{Ubuntu}, and \projname{FreeBSD}.
On each \gls{os}, we sequentially deploy different versions of \cp.
The combination of \gls{cpu}, \gls{os}, and version of \cp is a configuration.
In our experiment, we investigate 40 different configurations.
For each configuration, we execute \gls{pyperformance} and measure execution times of various benchmarks and their power draw during execution.

We find that all three variables of \gls{cpu}, \gls{os}, and version of \cp have a statistically significant impact on energy consumption during benchmark execution.
% Fill in high level results here!


The contributions of this paper are:
\begin{itemize}
  \item We present an experiment design to directly measure the energy consumption of 40 different configurations of \gls{cpu}, \gls{os}, and version of \cp.
  \item We demonstrate that benchmarks executed on certain configurations of \gls{cpu}, \gls{os}, and version of \cp consume significantly less energy compared to other widely used configurations.
  \item We provide a replication kit\footnote{\url{https://github.com/oskarbreindahl/Python_Application_Energy_Consumption}} that allows to automatically replicate our results. It also contains all results reported in this paper and allows for reproduction.\todo{Move repo and anonymize link for review.}
\end{itemize}
