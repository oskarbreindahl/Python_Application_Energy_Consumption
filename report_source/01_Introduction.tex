\newpage
\fancyhead[R]{Python Application Energy Consumption}
\section{Introduction}
\label{sec:introduction}

Python is currently being reported by some sources as the most popular programming language in the world\cite{pypl,tiobe}, while other sources place it among the top three most used\cite{stackover, statista}. Python is observed to have low performance, i.e. high runtime, compared to other languages, and consumes more energy\cite{pereira_rank_efficiency}. While the developers of Python work to improve the performance of the language, the Python Enhancement Proposal (PEP) index currently contains no proposal to lower energy consumption\cite{pep_index}. Meanwhile, the environmental impact of software is becoming more apparent, and software developers have an increasing responsibility for producing sustainable, energy-efficient software\cite{caballar2024we}. In this thesis, we present an experiment design for comparing the energy consumption among configurations of hardware, Python version, and OS, along with initial results from running said experiment. The primary goal of our efforts is to investigate whether choice of Python version, OS, and hardware is significant to energy consumption when running application benchmarks. To that end, we formulate the following research questions:
\begin{itemize}[label={}]
    \item \textbf{RQ1:} 
    \textit{What is the impact of choice of OS on energy consumption when running application benchmarks?}
    \item \textbf{RQ2:} 
    \textit{What is the impact of choice of Python version on energy consumption when running application benchmarks?}
    \item \textbf{RQ3:}
    \textit{What is the impact of choice of hardware on energy consumption when running application benchmarks?}
\end{itemize}
To simulate application execution, we use \textit{Pyperformance}, an open-source benchmarking tool used by the developers of Python. To measure energy consumption, we use a controller PC connected to a power meter known as the Otii Ace Pro, which is connected to a single-board computer known as a Raspberry Pi running multiple Pyperformance benchmarks. To provide a useful basis for comparison, benchmarks are run on four different operating systems with five different versions of Python on two different Raspberry Pis. After running the experiments, we find that choice of OS, Python version, and hardware all have a statistically significant impact in terms of energy consumption according to multiple statistical tests.

Our contributions are:
\begin{itemize}[label={-}]
    \item We present an experiment design to directly measure the energy consumption of 40 different configurations of OS, Python version, and hardware. Our design can serve as a blueprint for similar sustainability experiments.
    \item We show that choice of hardware, software, and OS all have a statistically significant impact on application energy consumption.
    \item We provide a complete replication kit with automated experiment provided with this thesis, found at \url{https://github.com/oskarbreindahl/Python_Application_Energy_Consumption}, henceforth referred to as \textit{the repository}.
\end{itemize}
\newpage
