\fancyhead[R]{Python Application Energy Consumption}
\section{Conclusion}
\label{sec:conclusion}
In this thesis, we investigated how the choice of operating system, Python version, and hardware configuration impacts energy consumption when running application benchmarks. We designed an experiment measuring the energy consumption on Raspberry Pis running application benchmarks from the Pyperformance benchmarking suite. We measured the energy consumption across 40 configurations involving five Python versions, four operating systems, and two Raspberry Pi versions. We then analysed the results and confirmed their statistical significance. Our findings show that all three variables—OS, Python version, and hardware—have statistically significant impacts on energy efficiency. The configuration with the lowest average energy consumption identified was Python3.12 running on Manjaro on a Raspberry Pi 4 Model B.\\
\textbf{Impact for practitioners:} Our results suggest that developers and system maintainers can achieve reductions in energy consumption of up to 85\% simply by choosing lower energy-consuming runtime environments. In particular, we recommend upgrading beyond Python3.10 and using modern hardware such as the Raspberry Pi 4, where possible. However, we caution that hardware upgrades carry environmental and economic costs, and legacy constraints may limit opportunities for refactoring.\\
\textbf{Impact for the CPython developers:} Our findings support the notion that improvements in performance in Python since version 3.11 have also delivered benefits in terms of energy consumption. However, the apparent plateau in improvements from versions 3.11 through 3.13 suggests diminishing returns. We recommend that the Python maintainers include energy consumption benchmarking as part of their standard performance evaluation process, using a setup like the one presented in this thesis.\\
\textbf{Impact for researchers:} This thesis extends Pfeiffer’s work by introducing a scalable, replicable framework for benchmarking energy consumption across diverse OS and hardware combinations. Open questions remain regarding the causes behind the observed improvements, particularly the role of OS-level specifics and Python implementation details. Further research should investigate whether the observed improvements are rooted in Python’s memory management, system calls, or something else. We hope this thesis inspires further research into environmentally sustainable software engineering practices.
