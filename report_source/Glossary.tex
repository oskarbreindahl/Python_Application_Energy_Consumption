\fancyhead[R]{\ref{sec:glossary} Glossary}
\label{sec:glossary}
\section*{Glossary}
% Please add the following required packages to your document preamble:
% \usepackage[table,xcdraw]{xcolor}
% If you use beamer only pass "xcolor=table" option, i.e. \documentclass[xcolor=table]{beamer}
\begin{tabularx}{\linewidth}{|l|X|}
\hline
\rowcolor[HTML]{C0C0C0} 
{\color[HTML]{000000} Term} & Definition                                                                                                            \\ \hline
API &
  Application Programming Interface. A set of instructions that can be used to interact with an application, module or service. \\ \hline
\rowcolor[HTML]{CBCEFB} 
Bash                        & Bourne Again Shell. Bash is a Unix shell commonly found in MacOS and Linux distributions.                             \\ \hline
Class                       & A term from object-oriented programming, a class is an extensible code template for creating objects.                 \\ \hline
\rowcolor[HTML]{CBCEFB} 
Class Loader                & Class loaders are responsible for loading classes into a JVM.                                                         \\ \hline
Dynamic Analysis            & Testing and evaluation of software during runtime.                                                                    \\ \hline
\rowcolor[HTML]{CBCEFB} 
Environment                 & A configuration of hardware and software such as operating system, program version, installed tools, etc.             \\ \hline
Fernflower                  & A decompiler for Java binaries and class-files made for IntelliJ.                                                     \\ \hline
\rowcolor[HTML]{CBCEFB} 
GUI                         & Graphical User Interface. A graphical interface that users can interact with to perform functionalities of a program. \\ \hline
Heatcode                    & Our prototype tool for analyzing Java binaries.                                                                       \\ \hline
\rowcolor[HTML]{CBCEFB} 
IDE &
  Integrated Development Environment. An environment that lets developers write and run code, additionally providing supporting tools to automate the process. \\ \hline
IntelliJ                    & A popular IDE for Java development.                                                                                   \\ \hline
\rowcolor[HTML]{CBCEFB} 
JSON                        & JavaScript Object Notation. Human-readable, lightweight data-interchange format.                                      \\ \hline
\rowcolor[HTML]{FFFFFF} 
Java                        & A popular high-level object-oriented programming language that is architecture agnostic.                              \\ \hline
\rowcolor[HTML]{CBCEFB} 
Java binaries (.jar-files)  & A file format that aggregates Java class files and metadata in one file for distribution.                             \\ \hline
\rowcolor[HTML]{FFFFFF} 
JavaScript                  & Not to be confused with Java, a popular high-level programming language typically used for web development.           \\ \hline
\rowcolor[HTML]{CBCEFB} 
JDI                         & Java Debugging Interface. A high-level interface that allows accessing the state of a JVM during runtime.             \\ \hline
\rowcolor[HTML]{FFFFFF} 
JVM                         & Java Virtual Machine. A specification for implementing a system-agnostic virtual machine that programs can run on.    \\ \hline
\rowcolor[HTML]{CBCEFB} 
Lambda (expression) &
  An anonymous function not bound to an identifier. In Java, they're typically used to implement an interface with only one method. \\ \hline
\rowcolor[HTML]{FFFFFF} 
Library                     & A collection of resources such as methods and objects used by a program.                                              \\ \hline
\rowcolor[HTML]{CBCEFB} 
Node.js                     & A back-end JavaScript runtime environment, typically used for modern web servers.                                     \\ \hline
\rowcolor[HTML]{FFFFFF} 
Profiling                   & The practice of evaluating a program based on certain characteristics.                                                \\ \hline
\rowcolor[HTML]{CBCEFB} 
Program                     & A sequence or set of instructions for a computer to execute.                                                          \\ \hline
\rowcolor[HTML]{FFFFFF} 
React                       & A JavaScript library for building declarative web applications.                                                       \\ \hline
\rowcolor[HTML]{CBCEFB} 
Refactoring &
  The process of altering the internal structure of a software project without changing the external functionality. \\ \hline
\rowcolor[HTML]{FFFFFF} 
Runtime (life-cycle)        & The phase of a program in which the program is actively running.                                                      \\ \hline
\rowcolor[HTML]{CBCEFB} 
Serialization               & Translating data to a format that can be persisted or transported.                                                    \\ \hline
\rowcolor[HTML]{FFFFFF} 
Source Code                 & A collection of text written to be human-readable that can be translated into instructions a computer can run.        \\ \hline
\rowcolor[HTML]{CBCEFB} 
SourcePrinter               & Our Java program that aggregates decompiled source code and method calls.                                             \\ \hline
\rowcolor[HTML]{FFFFFF} 
Static Analysis             & Extracting information about a program from its source code or other artifacts.                                       \\ \hline
\rowcolor[HTML]{CBCEFB} 
TypeScript                  & A statically typed programming language that transpiles to JavaScript.                                                \\ \hline
\end{tabularx}
\newpage